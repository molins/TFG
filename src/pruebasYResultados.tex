\chapter{Pruebas y resultados\label{sec:pruebasYResultados}}

% IDEAS A VENDER:
% Generalizable y escalable. Podemos cambiar el modelo subyacente sin sufrir
% Meter el apartado de análisis de resultados
% Se han usado por 50 personas y ha aguantado
% Hablar de la base de datos. Del diseño y cómo se ha hecho
% Probado en entorno real

\section{Comparativa con otras soluciones}

\section{2013-2014: Grado en Historia}

Durante el curso académico 2013-2014 los 15 alumnos de la asignatura de ``Historia Antigua I'' en el Grado en Historia de la Universidad Autónoma de Madrid utilizaron e-valUAM como herramienta de estudio y de evaluación.

A lo largo del semestre, los alumnos tuvieron disponibles 3 cuestionarios de autoevaluación y 3 exámenes. La autoevaluación estaba disponible a los alumnos tantas veces como quisieran. De los 15 alumnos, 6 la utilizaron menos de 3 veces, mientras que otros 6 la utilizaron más de 15 veces. 

A lo largo del año, entre tres profesores crearon un total de 372 preguntas, respondiéndose cada una de media unas 12,386 veces.

Los alumnos 

\section{Test sobre conocimientos informáticos}

% Contar que aquí cambiamos el modelo


\section{Caso real. Asignatura en el Grado en Educación Infantil}

\subsection{Aprendizaje}
\subsection{Examen final}
\subsection{Trabajo final}