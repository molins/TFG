% Resumen en inglés
\chapter*{Abstract}

\begin{abstractEn}
The introduction of technologies in teaching-learning environments is a possibility that brings many promising hopes. The areas where technology can contribute to education are very wide, and so, this work is focused on the evaluation process. Based on the principles of the \textit{Computer Adaptive Tests} (\acrshort {CAT}) and \textit {Adaptive Educational Hypermedia} (\acrshort {AEH}), as a result of this project an online and adaptive system for asistence learning process is presented: e-valUAM.

The development of e-valUAM has aspired to achieve the following objectives:
\begin{enumerate*}[label=\alph*\upshape)]
\item allow teachers, in any discipline, to create adaptive online questionnaires that serve teachers and students to measure their knowledge in a way that challenges students;
\item avoid the need of having advanced knowledge in computer science to respond, or create questionnaires;
\item make available automatic analysis tools to facilitate teachers' work on identifying deficiencies in the evaluation process; and 
\item implement a robust, flexible system that properly abstracts the dynamics of evaluation, so it can be used in real environments.
\end{enumerate*}

Throughout the life of this project two functional and incremental prototypes have been created, which have been used during the academic years 2013/2014 and 2014/2015. To create both prototypes, a complete analysis of requirements has been conducted, which has collected the characteristics and needs that the system should cover. Then, the system has been designed, which includes the domain, user and adaptation models that have been implemented afterwards in e-valUAM.

Both prototypes have been successfully tested in real environments, which have encompassed diverse disciplines. Thanks to this tool, students have been available to perform self-assessment questionnaires that have provided them with early feedback. In addition, teachers have been able to create automatic correction tests in which er rors have been detected quickly. Moreover a group of researchers has been able to test new models of adaptive testing.
\end{abstractEn}

% Palabras clave en inglés
\begin{keywordsEn}
Computer Adaptive Tests, Adaptive Educational Hypermedia, Aadaptive Systems, Online Systems, Evaluation Methodologies, Teaching-Learning Environmentes.
\end{keywordsEn}

% Resumen en español
\chapter*{Resumen}

\begin{abstractEs}
La introducción de las tecnologías en los procesos de enseñanza-aprendizaje es una posibilidad que trae consigo muchas y prometedoras esperanzas. Las áreas en las que la tecnología puede aportar a la educación son muy amplias, por lo que este trabajo se centra en el proceso de evaluación. Basándose en los principios de los \textit{Computer Adaptive Tests} (\acrshort{CAT}) y la \textit{Adaptive Educational Hypermedia} (\acrshort{AEH}), como resultado de este Trabajo Fin de Grado se presenta un sistema de ayuda al aprendizaje online y adaptativo: e-valUAM.

Con la creación de e-valUAM se ha aspirado a alcanzar los siguientes objetivos: 
\begin{enumerate*}[label=\alph*\upshape)]
\item Permitir crear a profesores de cualquier disciplina cuestionarios online adaptativos que sirvan para que los profesores o los propios alumnos puedan medir sus conocimientos de una forma que resulte un reto justo a cada alumno; 
\item que no sea necesario disponer de un conocimiento avanzado en informática para responder o crear los cuestionarios;
\item poner a disposición del profesor herramientas automáticas de análisis que faciliten su labor de detección de deficiencias en el proceso de evaluación; y 
\item implementar un sistema robusto, flexible y que abstraiga correctamente las dinámicas de una evaluación, para que pueda ser utilizado en entornos reales.
\end{enumerate*}

A lo largo de la vida del proyecto se han creado dos prototipos funcionales e incrementales que han sido utilizados durante los cursos de 2013/2014 y 2014/2015. Para crear ambos prototipos se ha llevado a cabo un completo análisis de requisitos que ha recogido las características y las necesidades que debía cubrir el sistema. A continuación, se ha realizado un diseño en el que se han creado modelos de dominio, usuario y adaptación que después han sido implementados en e-valUAM. 

Ambos prototipos han sido probados con éxito en entornos reales que han englobado disciplinas dispares. Gracias a esta herramienta, los alumnos han tenido a su disposición cuestionarios de autoevaluación que les han aportado retroalimentación temprana, los profesores han podido crear exámenes de correción automática en los que han detectado rapidamente errores y, también, un conjunto de investigadores han podido probar nuevos modelos de cuestionarios adaptativos.

\end{abstractEs}

% Palabras clave en español
\begin{keywordsEs}
Computer Adaptive Tests, Adaptive Educational Hypermedia, Sistemas Adaptativos, Sistemas Online, Metodologías de Evaluación, Entornos Enseñanza-Aprendizaje.
\end{keywordsEs}