\chapter{Introducción \label{sec:introduccion}}

En esta sección se detallará qué ha motivado la realización de este Trabajo Fin de Grado. A continuación, se explicará en qué marco se ha llevado a cabo, así como el alcance del proyecto, especificando sus objetivos. Por último, se expondrá  la estructura que sigue el resto del presente documento.

\section{Motivación}

%\textbf{¿Por qué son necesarios los test adaptativos?
%¿Qué puede aportar un sistema informático a los test adaptativos?
%¿Dónde se pueden utilizar?
%Listar ejemplos: MOOCs, AEH, educación clásica, contextos donde el usuario no dispone de conocimiento informático.}

La revolución que ha supuesto la introducción de las tecnologías informáticas en el día a día es una revolución de un profundo calado. La informática ha traído consigo mejoras inconmensurables en las comunicaciones, la automatización o el desarrollo científico que, en general, han permitido al ser humano librarse de tareas repetitivas y dedicar más esfuerzo a tareas verdaderamente interesantes.

La educación ha bebido de los avances en informática pero a un ritmo mucho menor que otras áreas, a pesar de ser uno de los pilares sociales. Aunque cada día es más habitual el aprendizaje asistido por tecnologías (\acrshort{TEL}) con el uso de ordenadores en las aulas, Internet como recurso docente o la introducción de pizarras digitales, \textbf{gran parte de la actividad educativa ha permanecido inalterada, anclada en modelos tradicionales}. Por ejemplo, en muchos casos la evaluación de los alumnos a día de hoy se sigue basando principalmente en cuestionarios creados cada año, que los alumnos responden en papel y los profesores corrigen a mano, uno a uno.

\textbf{Pero, la evaluación tradicional de los alumnos puede conllevar problemas}. Por un lado, al crearse cada año un nuevo conjunto de preguntas es posible que haya alguna pregunta mal planteada que no se descubra hasta la corrección. Así mismo, la cantidad ingente de tiempo que dedican los profesores en crear y corregir los exámenes es tiempo que dedican en tareas ingratas y repetitivas que podrían ser eliminadas. Ese tiempo también se traduce en que los alumnos sufran retrasos a la hora de recibir retroalimentación sobre su empeño, una información que es muy valiosa y de la que cuanto antes dispongan, mejor.

La aparición recientemente de nuevas líneas de investigación en educación y tecnologías, como los \acrshort{MOOC} o los \acrshort{SPOC}, demuestran que \textbf{existe interés en el tema} de la aplicación de la tecnología a entornos de enseñanza-aprendizaje.

De todos esos posibles campos donde la educación y las tecnologías pueden encontrarse, \textbf{este trabajo se centra en la parte de la evaluación del conocimiento}, como herramienta del estudiante como del docente para medir el desempeño del alumno. En ambos casos, esa información puede resultar clave. Hoy en día, aún en algunos casos, el alumno obtiene retroalimentación solo una vez por asignatura y justo al final. En otros casos, el alumno tendrá varios exámenes parciales, que le darán una información muy valiosa, pero a costa de un mayor esfuerzo del profesor.

Creyendo que la tecnología puede aportar mucho al proceso de evaluación, en este trabajo \textbf{se presenta una propuesta de un sistema adaptativo orientado a crear cuestionarios que sirvan de apoyo al aprendizaje, ya sea como sistema de autoevaluación de los alumnos o sistema de evaluación para los docentes.} 
Buscando que sea útil en la mayor cantidad de áreas del conocimiento y  contextos posibles, se ha creado para que sea fácil de usar, sin que los usuarios necesiten disponer de los conocimientos informáticos especiales. 

Así mismo, buscando que ahorre tiempo al profesorado, se ha creado un sistema que permite fácilmente crear nuevos cuestionarios, que se evalúen automáticamente y que vayan acompañados de un análisis de los resultados automático. Con ello se busca también que el docente pueda detectar lo más rápido posible las lagunas en sus alumnos y mejorar así ambos en el proceso enseñanza-aprendizaje.


%Algunas nuevas invenciones, como los \acrshort{MOOC} demuestran que el aprendizaje asistido por tecnologías, \acrshort{TEL}, es un área donde aún queda trabajo y en múltiples facetas. Este Trabajo 

%Desde su creación, los ordenadores han sido introducidos de forma progresiva en cada vez más sectores, con grandes beneficios. La educación es un ejemplo de ello, aunque aún todavía es un ejemplo incompleto. El aprendizaje asistido por tecnologías, \acrshort{TEL}, y en concreto, el aprendizaje asistido por ordenador, \acrshort{CAL}, es cada vez más habitual y ha sido aplicada con éxito a la educación presencial, semipresencial o a distancia, aportando grandes ventajas en cada modalidad. La reciente aparición y popularización de los \acrshort{MOOC} ha vuelto a demostrar la necesidad de seguir ampliando estas áreas {CITA}.

%Dentro de las \acrshort{CAL} una de las ramas de interés es la conocida como tests adaptativos por ordenador, o \acrshort{CAT}. Los \acrshort{CAT} se han utilizado para múltiples propósitos, como puntuación instantánea \cite{Wainer00}, la mejora de  competencias lingüísticas \cite{Chapelle06} , identificación de estilos de aprendizaje \cite{Ortigosa10}, la habilidad matemática \cite{Klinkenberg11}, o la evaluación del estado de salud \cite{Revicki97}.

%\textbf{¿Cuál ha sido exactamente el trabajo? ¿Motivación? Creo que no ¿e-valUAM? Sí ¿Modelo de estudiantes? Sí ¿Protocolo para crear las preguntas? También ¿Exámenes con duda? Sí.}

% sirve para múltiples objetivos, como aumentar la motivación de los alumnos {CITA}, sus resultados académicos {CITA} o su }}

\section{Marco y antecedentes}

%En qué investigación se engloba el proyecto. ¿Citar financiación?

Durante el curso académico 2011/2012 empezó una colaboración entre miembros de la Escuela Politécnica Superior y la Facultad de Filosofía y Letras, ambas de la Universidad Autónoma de Madrid. Gracias a dicha colaboración, durante los cursos 11/12 y 12/13 se llevaron a cabo varias experiencias en las que alumnos de los Grados en Ingeniería Informática e Historia colaboraban en la creación de videojuegos, resultando en un método docente altamente motivador \cite{Sevilla12}\cite{Molins14Videogames}.

De estas primeras experiencias surgió un esfuerzo interdisciplinar enfocado en la introducción de las tecnologías en el proceso docente. Durante finales del curso 12/13 se decidió explorar nuevas posibilidades, esta vez centrándose en los cuestionarios adaptativos utilizados para la evaluación. En ese momento fue cuando surgió la línea de investigación en la que se enmarca este TFG. 

En este sentido, durante el verano de 2013 se empezaron a definir unos modelos iniciales de usuario y de adaptación, que se plasmarían en un primer prototipo del sistema que se desarrollo para realizar las primeras experiencias en una asignatura del Grado de Historio durante el curso 2013/2014.

Con la incorporación al estas experiencias de profesores de la Facultad de Formación de Profesorado y Educación, también de la UAM, se inició el desarrollo de un segundo prototipo que se utilizó, esta vez, en una asignatura del Grado en Magisterio en Educación Infantil, explorando nuevos modelos, nueva funcionalidad y en un entorno con más usuarios.

Esta investigación ha sido apoyada por la Convocatoria de Proyectos de Innovación Docente de la Universidad Autónoma de Madrid durante los cursos 2012/2013, 2013/2014, y 2014/2015, y por una Beca de Colaboración del Ministerio de Educación, Cultura y Deporte para el curso 2014/2015.

\section{Alcance y objetivos}

%¿Qué pretende lograr el sistema?
%¿Qué NO pretende lograr el sistema?

Dentro de la línea de investigación en la que se engloba el proyecto, \textbf{este Trabajo Fin de Grado no se ha centrado en la parte psicométrica, sino en la parte técnica del mismo}. Para probar los modelos planteados a lo largo de la investigación desde el principio se entendió como necesario crear un sistema online que implementara los modelos que se querían probar, para que pudiera ser utilizado en experiencias reales y de ahí obtener resultados que analizar. Así, este TFG tiene como sus objetivos:

\begin{itemize}
	\item Disponer de un sistema que \textbf{permita a profesores crear cuestionarios online} para que sean utilizados como herramienta de autoevaluación por los alumnos o como parte de una evaluación por parte del docente.
	\item Que sea un sistema que puedan utilizar profesores y alumnos de \textbf{múltiples áreas del conocimiento, sin que necesiten conocimientos informáticos avanzados.}
	\item Habilitar para el profesor \textbf{herramientas de análisis} que le permitan detectar rápidamente las deficiencias que puedan existir en las preguntas  elaboradas o en el conocimiento de los alumnos.
	\item Desarrollar un \textbf{sistema web robusto ante los picos de demanda} que suponen todos los alumnos de un curso accediendo a la vez durante un examen.
	\item Crear un \textbf{modelo de datos que abstraiga las entidades y relaciones de una evaluación}, que permita guardar toda la información posible para realizar análisis posteriores.
	\item Diseñar una \textbf{arquitectura para el sistema que sea flexible} y permita incorporar los nuevos modelos que se van a ir desarrollando con la investigación.
\end{itemize}

Algunos de los resultados parciales obtenidos hasta ahora han sido publicados en las siguientes comunicaciones o artículos:

\begin{itemize}
	\item González-Sacristán, C., Molins-Ruano, P., Díez, F., Rodríguez, P., & Sacha, G. M. (2013, November). Computer-assisted assessment with item classification for programming skills. In Proceedings of the First International Conference on Technological Ecosystem for Enhancing Multiculturality (pp. 111-117). ACM.
	\item Molins-Ruano, P., Borrego-Gallardo, F., Sevilla, C., Jurado, F., Rodriguez, P., & Sacha, G. M. (2014, November). Constructing quality test with e-valUAM. In Computers in Education (SIIE), 2014 International Symposium on (pp. 195-200). IEEE.
	\item Molins-Ruano, P., González-Sacristán, C., Díez, F., Rodriguez, P., & Sacha, G. M. (2014). Adaptive Model for Computer-Assisted Assessment in Programming Skills. arXiv preprint arXiv:1403.1465.
\end{itemize}

Por último, y puesto que el sistema requiere de autentificación para poder ser explorado, se ha creado una cuenta a tal propósito cuyo nombre de usuario y contraseña son \textit{TFG}. El sistema está disponible en \url{http://sacha.ii.uam/e-valUAM} para acceder como alumno y en \url{http://sacha.ii.uam/e-valUAM/profesor} para acceder como profesor. La cuenta tiene permisos tanto de alumno como de profesor, por lo que se pueden explorar libremente todo el sistema.

\section{Estructura del documento}

En el siguiente capítulo, sobre el estado del arte, se presenta una breve revisión de la evolución de los cuestionarios adaptativos por ordenador y de las dos fuentes principales de técnicas usadas en este trabajo: los \textit{Computer Adaptive Tests} y la \textit{Adaptive Educational Hypermedia}.

A continuación, se presenta un capítulo dedicado a las fases de análisis, diseño y desarrollo de la solución propuesta. En este capítulo se detalla la metodología de proyecto elegida, el análisis de requisitos (funcionales y no funcionales), además de la división del sistema en módulos y cómo se han implementado todos ellos.

En el cuarto capítulo, dedicado a las pruebas y resultados, se exponen los resultados obtenidos con cada uno de los dos prototipos desarrollados, explicando los entornos reales donde se probaron, además de cuál fue su desempeño.

El siguiente capítulo expone las conclusiones del trabajo además de unas notas sobre el trabajo futuro, seguido de la bibliografía y una serie de apéndices. % TODO: Detallar los apéndices.

% Cuatro líneas. Resumir todo el trabajo. Ir haciéndolo