\chapter{Introducción \label{sec:introduccion}}

En esta sección se detallará qué ha motivado la realización de este Trabajo Fin de Grado. A continuación, se explicará en qué marco se ha llevado a cabo, así como el alcance del proyecto, especificando sus objetivos y sus límites. Por último, se expondrá  la estructura que sigue el resto del presente documento.

\section{Motivación}

%\textbf{¿Por qué son necesarios los test adaptativos?
%¿Qué puede aportar un sistema informático a los test adaptativos?
%¿Dónde se pueden utilizar?
%Listar ejemplos: MOOCs, AEH, educación clásica, contextos donde el usuario no dispone de conocimiento informático.}

La revolución que ha supuesto la introducción de las tecnologías informáticas en cada día más aspectos de la vida humana es una revolución de un profundo calado. La informática ha traído consigo mejoras inconmensurables en las comunicaciones, la automatización o el desarrollo científico (por citar solo algunos ejemplos) que, en general, han permitido al ser humano librarse de tareas repetitivas y dedicar más esfuerzo a las tareas verdaderamente interesantes.

La educación, a pesar de ser uno de los pilares sociales, ha bebido de los avances en informática pero a un ritmo mucho menor que otras áreas. Aunque cada día es más habitual el aprendizaje asistido por tecnologías, \acrshort{TEL}, con el uso de ordenadores en las aulas, Internet como recurso docente o pizarras digitales en las aulas, gran parte de la actividad educativa ha permanecido inalterada, anclada en modelos artesanales. Por ejemplo, la evaluación de los alumnos a día de hoy se sigue basando principalmente en cuestionarios creados cada año, que los alumnos responden en papel y los profesores corrigen a mano, uno a uno.

El caso de la evaluación artesanal de los alumnos trae consigo varios problemas. Por un lado, al crearse cada año un nuevo examen es probable que algún año haya alguna pregunta mal planteada que no se descubra hasta la corrección. Así mismo, la cantidad ingente de tiempo que pierden los profesores es tiempo que no dedican a explicar a los alumnos temario, reforzar las partes más complicadas o a responder dudas. Ese tiempo también se traduce en que los alumnos sufran retrasos a la hora de tener retroalimentación sobre su empeño, una información que es muy valiosa y de la que cuanto antes dispongan, mejor.

La aparición recientemente de nuevas líneas de investigación en educación y tecnologías, como los \acrshort{MOOC} o los \acrshort{SPOC}, demuestran que existe interés en el tema y campos donde realizar mejoras.

De todos esos posibles campos donde la educación y las tecnologías pueden encontrarse, este trabajo se centra en la parte de la evaluación del conocimiento, tanto como herramienta del estudiante para medir su desempeño como herramienta del docente para conocer el estado de sus alumnos. En ambos casos, esa información puede resultar clave. Hoy en día, aún en muchos casos, el alumno obtiene retroalimentación solo una vez por asignatura y justo al final, cuando ya no puede rectificar. En el mejor de los casos, el alumno tendrá uno o dos exámenes parciales, que le darán algo de información antes de que sea demasiado tarde, pero será a costa de un mayor esfuerzo del profesor.

Creyendo que la tecnología puede aportar mucho al proceso de evaluación, en este trabajo se presenta una propuesta de un sistema adaptativo orientado a crear cuestionarios que sirvan de apoyo al aprendizaje, ya sea como sistema de autoevaluación de los alumnos o sistema de evaluación para los docentes. 
Buscando que sea útil en la mayor cantidad de contextos posibles, se ha creado para que sea fácil de usar, para todas las edades y todos los niveles de conocimiento informático. Así mismo, buscando que ahorre tiempo al profesorado, se ha creado un sistema que permite fácilmente crear nuevos cuestionarios una vez confeccionada una batería de preguntas, que se evalúan automáticamente y que van acompañados de un análisis de los resultados automático, para que el docente pueda detectar lo más rápido posible las lagunas en sus alumnos.


%Algunas nuevas invenciones, como los \acrshort{MOOC} demuestran que el aprendizaje asistido por tecnologías, \acrshort{TEL}, es un área donde aún queda trabajo y en múltiples facetas. Este Trabajo 

%Desde su creación, los ordenadores han sido introducidos de forma progresiva en cada vez más sectores, con grandes beneficios. La educación es un ejemplo de ello, aunque aún todavía es un ejemplo incompleto. El aprendizaje asistido por tecnologías, \acrshort{TEL}, y en concreto, el aprendizaje asistido por ordenador, \acrshort{CAL}, es cada vez más habitual y ha sido aplicada con éxito a la educación presencial, semipresencial o a distancia, aportando grandes ventajas en cada modalidad. La reciente aparición y popularización de los \acrshort{MOOC} ha vuelto a demostrar la necesidad de seguir ampliando estas áreas {CITA}.

%Dentro de las \acrshort{CAL} una de las ramas de interés es la conocida como tests adaptativos por ordenador, o \acrshort{CAT}. Los \acrshort{CAT} se han utilizado para múltiples propósitos, como puntuación instantánea \cite{Wainer00}, la mejora de  competencias lingüísticas \cite{Chapelle06} , identificación de estilos de aprendizaje \cite{Ortigosa10}, la habilidad matemática \cite{Klinkenberg11}, o la evaluación del estado de salud \cite{Revicki97}.

%\textbf{¿Cuál ha sido exactamente el trabajo? ¿Motivación? Creo que no ¿e-valUAM? Sí ¿Modelo de estudiantes? Sí ¿Protocolo para crear las preguntas? También ¿Exámenes con duda? Sí.}

% sirve para múltiples objetivos, como aumentar la motivación de los alumnos {CITA}, sus resultados académicos {CITA} o su }}

\section{Marco}

%En qué investigación se engloba el proyecto. ¿Citar financiación?

Este TFG se corresponde con la parte técnica llevada a cabo dentro de una línea de investigación desarrollada en el Dpto. de Ingeniería Informática de la Universidad Autónoma de Madrid. La investigación, liderada por el Dr. Sacha Gómez Moviñas y la Dra. Pilar Rodríguez Marín, tiene como objetivo explorar nuevos modelos % Y mucho más... ¡Ni te imaginas todo lo que quiere hacer!

\section{Alcance y objetivos}

%¿Qué pretende lograr el sistema?
%¿Qué NO pretende lograr el sistema?

Dentro de la línea de investigación en la que se engloba el proyecto, este Trabajo Fin de Grado se ha centrado exclusivamente en las contribuciones realizadas en la parte técnica del mismo. Para probar las hipótesis planteadas durante la investigación desde el principio se entendió como necesario crear un sistema online que implementara los modelos que se querían probar, para que pudiera ser utilizado en experiencias reales y de ahí obtener resultados con los que analizar la propuesta. Así, este TFG tenía como sus objetivos:

\begin{itemize}
	\item Disponer de un sistema que permita a profesores crear cuestionarios online para que sean utilizados como herramienta de autoevaluación por los alumnos o como parte de una evaluación por parte del docente.
	\item Que profesores y alumnos, sin requerir altos conocimientos informáticos, puedan desenvolverse correctamente.
	\item Habilitar para el profesor herramientas de análisis que le permitan detectar rápidamente las deficiencias que pueda haber en las preguntas  elaboradas o en el conocimiento de los alumnos.
	\item Desarrollar un sistema web robusto ante los picos de demanda que suponen todos los alumnos de un curso accediendo a la vez.
	\item Crear un modelo de datos que abstrajera las entidades y relaciones más importantes en una evaluación, que permitiera guardar toda la información posible para realizar análisis posteriores.
	\item Diseñar una arquitectura para el sistema que fuera lo suficientemente flexible para permitir incorporar de forma sencilla los nuevos modelos que se iban a ir desarrollando conforme avanzara la investigación a la que iba asociada.
\end{itemize}

Aunque son objetivos finales del proyecto, los siguientes objetivos, por limitaciones de tiempo, no han sido cubiertos en el presente trabajo, o al menos, no completamente:

\begin{itemize}
	\item
\end{itemize}



\section{Estructura del documento}

TODO: Descripción de la estructura del documento

% Cuatro líneas. Resumir todo el trabajo. Ir haciéndolo