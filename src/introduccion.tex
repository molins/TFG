\chapter{Introducción \label{sec:introduccion}}

En esta sección se detallará qué ha motivado la realización de este Trabajo Fin de Grado. A continuación, se explicará en qué marco se ha llevado a cabo, así como el alcance del proyecto, especificando sus objetivos y sus límites. Por último, se expondrá  la estructura que sigue el resto del presente documento.

\section{Motivación}

\textbf{¿Por qué son necesarios los test adaptativos?
¿Qué puede aportar un sistema informático a los test adaptativos?
¿Dónde se pueden utilizar?
Listar ejemplos: MOOCs, educación clásica, contextos donde el usuario no dispone de conocimeinto informático.}

Desde su creación, los ordenadores han sido introducidos de forma progresiva en cada vez más sectores, con grandes beneficios. La educación es un ejemplo de ello, aunque aún todavía es un ejemplo incompleto. El aprendizaje asistido por tecnologías, \acrshort{TEL}, y en concreto, el aprendizaje asistido por ordenador, \acrshort{CAL}, 
es cada vez más habitual y ha sido aplicada con éxito a la educación presencial, semipresencial o a distancia, aportando grandes ventajas en cada modalidad. La reciente aparición y popularización de los \acrshort{MOOC} ha vuelto a demostrar la necesidad de seguir ampliando estas áreas {CITA}.

Dentro de las \acrshort{CAL} una de las ramas de interés es la conocida como tests adaptativos por ordenador, o \acrshort{CAT}. Los \acrshort{CAT} se han utilizado para múltiples propósitos, como puntuación instantánea \cite{Wainer00}, la mejora de  competencias lingüísticas \cite{Chapelle06} , identificación de estilos de aprendizaje \cite{Ortigosa10}, la habilidad matemática \cite{Klinkenberg11}, o la evaluación del estado de salud \cite{Revicki97}.

\textbf{¿Cuál ha sido exactamente el trabajo? ¿Motivación? Creo que no ¿e-valUAM? Sí ¿Modelo de estudiantes? Sí ¿Protocolo para crear las preguntas? También ¿Exámenes con duda? Sí.}

 sirve para múltiples objetivos, como aumentar la motivación de los alumnos {CITA}, sus resultados académicos {CITA} o su 

\section{Marco}

En qué investigación se engloba el proyecto. ¿Citar financiación?

\section{Alcance}

¿Qué pretende lograr el sistema?
¿Qué NO pretende lograr el sistema?

\section{Estructura del documento}

TODO: Descripción de la estructura del documento