% Acrónimos

% TODO: Añadir aquí los acrónimos
% Ejemplo de acrónimo
\newacronym{FPGA}{FPGA}{Field-Programmable Gate Array}
\newacronym{TEL}{TEL}{Aprendizaje asistido por tecnologías, por sus siglas en inglés, \textit{Technology-Enhanced Learning}}
\newacronym{MOOC}{MOOC}{Curso en línea, masivo y abierto; por sus siglas en inglés, \textit{Massive Open Online Course}}
\newacronym{CAL}{CAL}{Aprendizaje asistido por ordenador, por sus siglas en inglés, \textit{Computer-Aided Learning}}
\newacronym{CAT}{CAT}{Tests adaptativos por ordenador, por sus siglas en inglés, \textit{Computer Adaptive Tests}}
\newacronym{WCAG}{WCAG}{Pautas para la accesibilidad del contenido web, por sus siglas en inglés, \textit{Web Content Accessibility Guidelines}. Es un estándar creado por la W3C}
\newacronym{SPOC}{SPOC}{Curso online pequeño y privado, por sus siglas en inglés, \textit{Small Private Online Course}. Término acuñado para referirse a una versión de un \acrshort{MOOC} usado localmente con estudiantes presenciales.}
\newacronym{e-valUAM}{e-valUAM}{Nombre con el que se conoce al sistema adaptativo para ayuda al aprendizaje desarrollado como parte de este Trabajo Fin de Grado.}
\newacronym{IRT}{IRT}{\textit{Item Response Theory}}
\newacronym{AH}{AH}{\textit{Adaptive hypermedia}}
\newacronym{AEH}{AEH}{\textit{Adaptive Educational Hypermedia}}
\newacronym{CTT}{CTT}{\textit{Classical Test Theory}}

% Glosario

% TODO: Añadir aquí las definiciones del glosario
% Ejemplo de glosario
\newglossaryentry{bitstream}{name={bitstream},description={En este contexto se refiere al binario que configura el Hardware de la FPGA}}