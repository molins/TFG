% \chapter{Análisis de requisito ampliado\label{apen:analisis de requisitos}}

% \begin{rnf0}
% 	\item Accesibilidad
% 	\begin{rnf0*}
% 		\item El sistema debe cumplir con el estándar \acrshort{WCAG} 2.0 en un nivel A.
% 	\end{rnf0*}
% \end{rnf0}


\chapter{Estructura del proyecto\label{apen:estructura proyecto}}

\dirtree{%
 .1 /.
 .2 index.php\DTcomment{Pantalla de login}.
 .2 eleccionExamen.php\DTcomment{Presentación y listado de exámenes disponibles}.
 .2 examen.php\DTcomment{Fichero con el motor de adaptación}.
 .2 finExamen.php\DTcomment{Resultados del examen}.
 .2 cambiarContrasenya.php.
 .2 estilo.css.
 .2 favicon.png.
 .2 funcionesAlumno.php\DTcomment{Fichero auxiliar con funciones PHP}.
 .2 multimedia\DTcomment{Directorio para los ficheros multimedia}.
 .3 0\DTcomment{Un directorio por cada materia, con sus ficheros dentro}.
 .4 Amon.bmp.
 .4 Anubis.bmp.
 .4 Asedioramses.bmp.
 .4 \ldots.
 .3 1.
 .4 Archivoebla.bmp.
 .4 Codigodehammurabi.bmp.
 .4 \ldots.
 .3 \ldots.
 .3 46\DTcomment{En total se han creado 46 materias distintas}.
 .4 1.mp3.
 .4 1a.jpg.
 .4 1b.jpg.
 .4 1c.jpg.
 .4 2.mp3.
 .4 \ldots.
 .3 logos\DTcomment{Directorio con los logos corporativos}.
 .4 ope.bmp.
 .4 uam.jpg.
 .2 profesor\DTcomment{Directorio con los ficheros de la sección de profesor}.
 .3 index.php\DTcomment{Login}.
 .3 salir.php\DTcomment{Logout}.
 .3 ayuda.php\DTcomment{Introducción al sistema de gestión}.
 .3 profesor.php\DTcomment{Comprueba que el usuario tenga permisos de profesor}.
 .3 funcionesProfesor.php\DTcomment{Fichero auxiliar con funciones PHP}.
 .3 materiasGestion.php\DTcomment{\texttt{*Gestion} es la interfaz de creación y listado}.
 .3 materaiasRequest.php\DTcomment{\texttt{*Request} procesa peticiones AJAX de \texttt{*Gestion}}.
 .3 preguntasGestion.php\DTcomment{Hay un \texttt{*Gestion} y un \texttt{*Request} para cada ente}.
 .3 preguntasRequest.php.
 .3 preguntasMostrar.php\DTcomment{Interfaz que lista todas las preguntas de una materia}.
}

\dirtree{%
 .1 /.
 .2 profesor.
 .3 examenGestion.php.
 .3 examenRequest.php.
 .3 multimediaGestion.php.
 .3 multimediaRequest.php.
 .3 multimediaView.php\DTcomment{Interfaz que lista todos los ficheros de una materia}.
 .3 estadisticas.php\DTcomment{Módulo de análisis de resultados}.
 .3 estadisticasRequest.php.
 .3 recuperarExamenesRequest.php\DTcomment{Módulo para revisar los cuestionarios respondidos}.
 .3 recuperarExamenes.php.
 .2 vendor\DTcomment{Directorio con las bibliotecas usadas en el proyecto}.
 .3 bootstrap.min.css.
 .3 bootstrap.min.js.
 .3 boostrap-confirmation.js.
 .3 jquery.min.js.
}

%\chapter{Código más relevante\label{apen:codigo}}


%
%\chapter{Ejemplos de bloques y comandos útiles en LaTeX\label{sec:ejemplos}}
%\section{Ejemplo de sección}
%
%
% Breve guía de comandos útiles para la memoria
%
%
% Citar una referencia
%La DARPA creo el protocolo de Internet \cite{ipv4sta}.
%
% Citar un elemento del glosario
%Citamos el acrónimo \gls{FPGA}.
%
% Citar un elemento del glosario (primera letra en may´usculas)
%\Gls{bitstream} es una secuencia de bits.
%
% Insertar una imagen con pie de página
%\begin{figure}[htp!]
%  \centering
%  \includegraphics[width=0.75\textwidth,clip=true]{Logo_UAM}
%  \caption{Logo de la Universidad Autónoma de madrid.}
%  \label{fig:logo_uam}
%\end{figure} 
%
% Referenciar una etiqueta (label)
%La figura~\ref{fig:logo_uam} se utiliza en la portada.
%
% Nueva página
%\clearpage
%
% Añadir código fuente sin líneas
%\begin{lstlisting}[label=algoritmo:quicksort,language=C,frame=single,caption=Algoritmo de ordenación Quicksort]
%#include <stdio.h>
% 
%void quick_sort (int *a, int n) {
%    int i, j, p, t;
%    if (n < 2)
%        return;
%    p = a[n / 2];
%    for (i = 0, j = n - 1;; i++, j--) {
%        while (a[i] < p)
%            i++;
%        while (p < a[j])
%            j--;
%        if (i >= j)
%            break;
%        t = a[i];
%        a[i] = a[j];
%        a[j] = t;
%    }
%    quick_sort(a, i);
%    quick_sort(a + i, n - i);
%}
%\end{lstlisting}
%
% Bloque de código inseparable
%\begin{code}
%#include <stdio.h>
% 
%void quick_sort (int *a, int n) {
%    int i, j, p, t;
%    if (n < 2)
%        return;
%    p = a[n / 2];
%    for (i = 0, j = n - 1;; i++, j--) {
%        while (a[i] < p)
%            i++;
%        while (p < a[j])
%            j--;
%        if (i >= j)
%            break;
%        t = a[i];
%        a[i] = a[j];
%        a[j] = t;
%    }
%    quick_sort(a, i);
%    quick_sort(a + i, n - i);
%}
%\end{code}
%
% Fórmula dentro de una línea de texto
%La ecuación de Euler ($e^{ \pm i\theta } = \cos \theta \pm i\sin \theta$) es citada frecuentemente como un ejemplo de belleza matemática.
%
% Fórmula independiente
%\begin{equation}\label{eq:pythagoras}
%a^2 + b^2 = c^2
%\end{equation}
