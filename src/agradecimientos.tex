\chapter*{Agradecimientos}
~

Este trabajo no hubiera sido posible sin que hace tres años Sacha y Pilar me dejaran jugar con ellos. Quiero creer que gracias a ello ahora soy un poco menos ``yogurín''. Gracias a Sacha por brindarme esta oportunidad y haberme empujado hacia adelante estos años y a Pilar por su impagable consejo, ayuda y paciencia.

También quiero dar las gracias a Covadonga Sevilla, Francisco L. Borrego, Santiago Atrio y Simone Santini por haber accedido a utilizar e-valUAM en sus clases. Así mismo, gracias (o perdón) a todos los alumnos de las asignaturas de ``Historia Antigua I'' del Grado en Historia, ``El Entorno como Instrumento Educativo'' del Grado en Educación Infantil, ``Proyecto de Programación'' del Doble Grado en Informática y Matemáticas y ``Diseño y Análisis de Algoritmos'' del Grado y Doble Grado en Ingeniería Informática.

Gracias a mis compañeros de carrera, con los que estos cuatro años se han hecho mucho más divertidos: Álvaro, Dani, Diego, Euler, Isa, José, Julián, María, Miguel, Mónica y Rober. Gracias a los compañeros que traje de antes, aún conservo y tanto me han acompañado. Gracias también a todos los grandes profesores con los que he tenido el placer de aprender estos cuatro años.

Por último, gracias a mis padres y a mi increíble hermana Lucía.



% Cita
\begin{flushright}
\textit{%<<I often liked to play tricks on people when I was at MIT. One time, in mechanical drawing class, some joker picked up a French curve (a piece of plastic for drawing somooth curves--–a curly, funny-looking thing) and said ``I wonder if the curves on this thing have some special formula?''\\
%I thought for a moment and said ``Sure they do. The curces are very special curves. Lemme show ya,'' and I picked up my French curve and began to turn it slowly. ``The French curve is mado so that at the lowest point on each curve, no matter how you turn it, the tangent is horizontal.''\\
%All the guys in the class were holding their French curve up at different angles, holding their pencil up to it at the lowest point and laying it along, and discovering that, sure enough, the tangent is horizontal. They were all excited by this ``discovery''---even though they hal already gone through a certain amount of calculus and had already ``learned'' that the derivate (tangent) of the minumen (the lowest point) of \emph{any} curve is zero (horizontal). They didn't put two and two together. They didn't even know what they ``knew.''\\
\\<<I don`t know what's the matter with people: they don't learn by understanding; they learn by some other way---by rote or something. Their knowledge is so fragile!>>\\}
Richard P. Feynman. ``Surely You're Joking, Mr. Feynman!''
\end{flushright}

