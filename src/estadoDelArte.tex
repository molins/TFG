\chapter{Estado del Arte\label{sec:estado_del_arte}}

A lo largo de toda la historia de la evaluación se ha buscado el equilibrio entre exámenes individuales y colectivos. En un examen individual se puede lograr una batería de preguntas en las que no haya ninguna inapropiada y asegurar que el evaluado entiende correctamente la tarea. En los exámenes colectivos se asegura que la evaluación ha sido uniforme para todos los evaluados, a la vez que se reduce enormemente el coste de evaluar \cite{Wainer00}.

Durante la década de los setenta aparecieron los primeros trabajos que exploraban la posibilidad de crear exámenes administrados masivamente pero adaptados a los individuos, eligiendo las nuevas preguntas en función de las respuestas anteriores, con el objetivo de buscar la manera óptima de evaluar al individuo en cuestión \cite{Lord68}. Desde un principio fue evidente que este nuevo tipo de evaluación solo sería posible gracias a los ordenadores, por lo que se pasó a conocer como \textit{Computerized Adaptive Testing}, o \acrshort{CAT}.

Entre las mejoras que prometían los \acrshort{CAT}\cite{Green83}:

\begin{enumerate}
	\item Aumentan la seguridad de los test, ya que al estar disponibles...
	\item \textbf{Página 11 del libro de Wainer.}
\end{enumerate}

Los \acrshort{CAT} se han utilizado con éxito para múltiples propósitos, como puntuación instantánea \cite{Wainer00}, la mejora de competencias lingüísticas \cite{Chapelle06} , identificación de estilos de aprendizaje \cite{Ortigosa10}, la habilidad matemática \cite{Klinkenberg11}, o la evaluación del estado de salud \cite{Revicki97}.


In that framework, CAT allows seeing the
students as individuals, taking their own characteristics into
account. Typically, CAT systems are able to adapt the items
presented to the student depending on their former answers, often
including some kind of personalized feedback [12].
Though presenting evident advantages, CAT systems also face
some clear difficulties, been one of them the need of precalibrating
the items of the test before they can be used in real
situations. This implies that the test should be performed by a
large sample of the population in order to assume a good test
calibration [5].


Constructing quality test with e-valUAM

Technology-Enhanced Learning (TEL) has a growing presence in educational institutions by making use of Computer-Aided Learning (CAL): a keynote practice where teachers and students are assisted by computers in their teaching/learning processes. Under this perspective, the assessment process is one of those issues that can get profit from the use of these practices. Moreover, completely on-line learning environments, like the emergent trends in Massive Open Online Courses (MOOCs) [1], demonstrate the importance of using CAL environments.

In this context, learning systems, which provide any kind of immediate feedback to students, look to be more effective than traditional learning strategies [2]. We can find in the literature several examples of assessment tools for some specific learning domains that can give useful and interactive advices to the students [3] [4] [5] [6]. In the context of providing feedback and advice to the students for widespread domains, Computer Adaptive Testing (CAT) provides advantages apart from an instantaneous scoring [7]. 

In spite of the advantages of this methodology, items can be inadequately developed by being wrongly classified on a topic or not clearly written up, which can lead to poor quality tests. This situation leads to face an important issue: the need of pre-calibrating test items before they can be used. Usually, a good calibration process implies that the test should be performed by a large population sample [12]. Nevertheless, since the pre-calibration is not always feasible, student models are applied in order to refine the quality of the assessment strategy [19] [20] [21] [22].