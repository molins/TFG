\chapter{Estado del Arte\label{sec:estado_del_arte}}

TODO: Estado del arte

DEJAR PARA EL FINAL

AEH
MOOC




Computer-assisted assessment with item classification
for programming skills

The use of computers has been extended to almost any discipline
nowadays. Being essential for the students in the performance of
their future professional activities, it is also more than adequate to
include the use of computers in any dimension of their learning
process. One of the topics that can be strongly improved by
computer-assisted tools is the assessment process. It is apparent
that, at least for certain purposes, computer-based testing (CBT)
can offer distinct advantages, apart from instantaneous scoring
[1]. Within CBT, Computer Adaptive Testing (CAT) has been
used for very different purposes; such as language proficiency
improvement [2], learning styles identification [3], measurement
of chess playing proficiency [4] or Maths ability [5], improvement
of the efficiency of personality [6] or heath status assessment [7].
Different versions of CAT have been used in learning
environments for decades in order to personalize the learning
process [8][9][10][11]. In that framework, CAT allows seeing the
students as individuals, taking their own characteristics into
account. Typically, CAT systems are able to adapt the items
presented to the student depending on their former answers, often
including some kind of personalized feedback [12].
Though presenting evident advantages, CAT systems also face
some clear difficulties, been one of them the need of precalibrating
the items of the test before they can be used in real
situations. This implies that the test should be performed by a
large sample of the population in order to assume a good test
calibration [5]. However, such “real” pre-calibration is not always
feasible, what has led to a huge interest in the development of
different kinds of students’ models which could help in the precalibration
task [13][14][15][16].

Adaptive Model for Computer-Assisted Assessment in Programming Skills

Being essential for the students’ performance of their potential professional activities, it is more than adequate to consider the
use of computers in any dimension of their learning process. One of the topics that can be strongly improved by computerassisted
tools is the assessment process. In this context, Computer-based testing (CBT) can offer distinct advantages. Within
CBT, Computer Adaptive Testing (CAT) has been used for very different purposes such as instantaneous scoring [1],
language proficiency improvement [2], learning styles identification [3], measurement of chess playing proficiency [4], Maths
ability [5], improvement of the efficiency of personality [6], or heath status assessment [7]. Different versions of CAT have
been used in learning environments in order to personalize the learning process [8] [9] [10] [11]. In that framework, CAT
allows seeing the students as individuals, taking their own characteristics into account. Typically, CAT systems are able to
adapt the items presented to the student depending on their former answers, often including some kind of personalized
feedback [12].
One of the most important difficulties in CAT systems is the need of pre-calibrating the items before they are used in real
assessments. For this reason, the test should be performed by a large sample of the population in order to get a good test
calibration [5]. The complexity of pre-calibration has led to a huge interest in the development of students’ models which
could help in the pre-calibration task [13] [14] [15] [16].


Constructing quality test with e-valUAM

Technology-Enhanced Learning (TEL) has a growing presence in educational institutions by making use of Computer-Aided Learning (CAL): a keynote practice where teachers and students are assisted by computers in their teaching/learning processes. Under this perspective, the assessment process is one of those issues that can get profit from the use of these practices. Moreover, completely on-line learning environments, like the emergent trends in Massive Open Online Courses (MOOCs) [1], demonstrate the importance of using CAL environments.

In this context, learning systems, which provide any kind of immediate feedback to students, look to be more effective than traditional learning strategies [2]. We can find in the literature several examples of assessment tools for some specific learning domains that can give useful and interactive advices to the students [3] [4] [5] [6]. In the context of providing feedback and advice to the students for widespread domains, Computer Adaptive Testing (CAT) provides advantages apart from an instantaneous scoring [7]. CAT takes the students' characteristics into account in order to be able to adapt the items depending on their former answers, often including some kind of personalized feedback [8]. CAT applications can be found in language proficiency improvement [9], learning styles identification [10], measurement of chess playing proficiency [11], Maths ability [12], and improvement of the efficiency of personality [13] or heath status assessment [14]. Different CAT approaches has been also used in learning environments for decades adapting the learning process [15] [16] [17] [18].

In spite of the advantages of this methodology, items can be inadequately developed by being wrongly classified on a topic or not clearly written up, which can lead to poor quality tests. This situation leads to face an important issue: the need of pre-calibrating test items before they can be used. Usually, a good calibration process implies that the test should be performed by a large population sample [12]. Nevertheless, since the pre-calibration is not always feasible, student models are applied in order to refine the quality of the assessment strategy [19] [20] [21] [22].