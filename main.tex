% Definiciones y constantes de estilo
% Clase del documento
\documentclass[a4paper,12pt,twoside,openright,titlepage]{book}

%
% Paquetes necesarios
%

% Símbolo del euro
\usepackage{eurosym}
% Codificación UTF8
\usepackage[utf8]{inputenc}
% Caracteres del español
\usepackage[spanish]{babel}
% Código, algoritmos, etc.
\usepackage{listings}
% Definición de colores
\usepackage{color}
% Extensión del paquete color
\usepackage[table,xcdraw]{xcolor}
% Márgenes
\usepackage{anysize}
% Cabecera y pie de página
\usepackage{fancyhdr}
% Estilo título capítulos
%\usepackage{quotchap}
% Algoritmos (expresarlos mejor)
\usepackage{algorithmic}
% Citas mejoradas
\usepackage{cite}
% Títulos de secciones
\usepackage{titlesec}
% Fórmulas matemáticas
\usepackage[cmex10]{amsmath}
% Enumeraciones
\usepackage{enumerate}
%\usepackage{enumitem}
\usepackage[inline]{enumitem}
% Páginas en blanco
\usepackage{emptypage}
% Separación entre cajas
\usepackage{float}
% Imágenes
\usepackage[pdftex]{graphicx}
% Mejora de las tablas
\usepackage{array}
% Mejora de los símbolos matemáticos
\usepackage{mdwmath}
% Separar figuras en subfiguras
\usepackage[caption=false,font=footnotesize]{subfig}
% Incluir pdfs externos
\usepackage{pdfpages}
% Mejoras sobre las cajas
\usepackage{fancybox}
% Apéndices
\usepackage{appendix}
% Marcadores (para el pdf)
\usepackage{bookmark}
% Estilo de enumeraciones
\usepackage{enumitem}
% Espacio entre líneas y párrafos
\usepackage{setspace}
% Glosario/Acrónimos
\usepackage[toc,xindy]{glossaries}
% Para evitar mal numeración de citas
\usepackage{notoccite}
% Para el árbol de directorio
\usepackage{dirtree}

%%%% Contribuciones propias

% Títulos de capítulo 
% http://osl.ugr.es/CTAN/macros/latex/contrib/titlesec/titlesec.pdf
\titleformat{\chapter}
{\normalfont\LARGE\bfseries}{\thechapter}{1em}{}

\titleformat{\section}
{\normalfont\Large\bfseries}{\thesection}{1em}{}

\titleformat{\subsection}
{\normalfont\large\bfseries}{\thesubsection}{1em}{}

% http://tex.stackexchange.com/questions/53338/reducing-spacing-after-headings
\titlespacing{\chapter}{0pt}{8pt plus 2pt minus 2pt}{4pt plus 2pt minus 2pt}
\titlespacing{\section}{0pt}{6pt plus 2pt minus 2pt}{2pt plus 2pt minus 2pt}
\titlespacing{\subsection}{0pt}{4pt plus 2pt minus 2pt}{0pt plus 2pt minus 2pt}


% Listas de requisitos
% Funcionales
\newenvironment{rf0} % Comando primer nivel
{ \begin{enumerate}[label=\bfseries RF \arabic*.] }
{ \end{enumerate} }

\newenvironment{rf0*} % Comando cualquier nivel > 0
{ \begin{enumerate}[label*=\bfseries\arabic*.] }
{ \end{enumerate} }


% No funcionales
\newenvironment{rnf0} % Comando primer nivel
{ \begin{enumerate}[label=\bfseries RNF \arabic*.] }
{ \end{enumerate} }

\newenvironment{rnf0*} % Comando cualquier nivel > 0
{ \begin{enumerate}[label*=\bfseries\arabic*.] }
{ \end{enumerate} }


%%%% Sacado de la plantilla

% Enlaces
\hypersetup{hidelinks,pageanchor=false,colorlinks,citecolor=Fuchsia,urlcolor=black,linkcolor=Cerulean}

% Euro (€)
\DeclareUnicodeCharacter{20AC}{\euro}

% Estilo de la bibliografía
\bibliographystyle{IEEEtran}

% Inclusión de gráficos
\graphicspath{{./graphics/}}

% Extensiones de gráficos
\DeclareGraphicsExtensions{.pdf,.jpeg,.jpg,.png}

% Definiciones de colores (para hidelinks)
\definecolor{LightCyan}{rgb}{0,0,0}
\definecolor{Cerulean}{rgb}{0,0,0}
\definecolor{Fuchsia}{rgb}{0,0,0}

% Keywords (español e inglés)
\def\keywordsEn{\vspace{.5em}
{\textbf{\textit{Key words ---}}\,\relax%
}}
\def\endkeywordsEn{\par}

\def\keywordsEs{\vspace{.5em}
{\textbf{\textit{Palabras clave ---}}\,\relax%
}}
\def\endkeywordsEs{\par}


% Abstract (español e inglés)
\def\abstractEs{\vspace{.5em}
{\textbf{\textit{Resumen ---}}\,\relax%
}}
\def\endabstractEs{\par}

\def\abstractEn{\vspace{.5em}
{\textbf{\textit{Abstract ---}}\,\relax%
}}
\def\endabstractEn{\par}

% Estilo páginas de capítulos
\fancypagestyle{plain}{
\fancyhf{}
\fancyfoot[CO]{\footnotesize\emph{\nombretrabajo}}
\fancyfoot[RO]{\thepage}
\renewcommand{\footrulewidth}{.6pt}
\renewcommand{\headrulewidth}{0.0pt}
}

% Estilo resto de páginas
\pagestyle{fancy}

% Estilo páginas impares
\fancyfoot[CO]{\footnotesize\emph{\nombretrabajo}}
\fancyfoot[RO]{\thepage}
\rhead[]{}

% Estilo páginas pares
\fancyfoot[CE]{\emph{\pieparcen}}
\fancyfoot[LE]{\thepage}
\fancyfoot[RE]{\pieparizq}
\lhead[]{}

% Guía del pie de página
\renewcommand{\footrulewidth}{.6pt}

% Nombre de los bloques de código
\renewcommand{\lstlistingname}{Código}

% Estilo de los lstlistings
\lstset{
    frame=tb,
    breaklines=true,
    postbreak=\raisebox{0ex}[0ex][0ex]{\ensuremath{\color{gray}\hookrightarrow\space}}
}

% Definiciones de funciones para los títulos
\newlength\salto
\setlength{\salto}{3.5ex plus 1ex minus .2ex}
\newlength\resalto
\setlength{\resalto}{2.3ex plus.2ex}

% Estilo de los acrónimos
\renewcommand{\acronymname}{Glosario}
\renewcommand{\glossaryname}{Glosario}
\pretolerance=2000
\tolerance=3000

% Pie de tabla
\addto\captionsspanish{
\def\tablename{Tabla}
\def\listtablename{\'Indice de tablas}
}

% Traducir appendix/appendices
\renewcommand\appendixtocname{Apéndices}
\renewcommand\appendixpagename{Apéndices}

% Comando code (lstlisting sin cambio de página)
\lstnewenvironment{code}[1][]%
  { \noindent\minipage{0.935\linewidth}\medskip
    \vspace{5mm}
    \lstset{basicstyle=\ttfamily\footnotesize,#1}}
  {\endminipage}

% Definiciones de comandos
\newcommand{\nombreautor}{Pablo Molins Ruano}
\newcommand{\nombretutor}{Pilar Rodríguez Marin}

% Descomentar si tu trabajo tiene un ponente
%\newcommand{\nombreponente}{TODO: Nombre del ponente}

\newcommand{\nombretrabajo}{Desarrollo de un sistema de cuestionarios adaptativos para el apoyo al aprendizaje}
\newcommand{\fecha}{Junio 2015}
\newcommand{\grado}{Grado en Ingeniería Informática}
\newcommand{\facultad}{Escuela Politécnica Superior}
\newcommand{\universidad}{Universidad Autónoma de Madrid}
\newcommand{\pieparizq}{}
\newcommand{\pieparcen}{}
\newcommand{\logoizq}{Logo_EPS}
\newcommand{\logoder}{Logo_UAM}
\newcommand{\correo}{pablo.molins@estudiante.uam.es}

% Glosario y acrónimos
\glossarystyle{altlist}
\makeglossaries
% Acrónimos

% TODO: Añadir aquí los acrónimos
% Ejemplo de acrónimo
\newacronym{FPGA}{FPGA}{Field-Programmable Gate Array}
\newacronym{TEL}{TEL}{Aprendizaje asistido por tecnologías, por sus siglas en inglés, \textit{Technology-Enhanced Learning}}
\newacronym{MOOC}{MOOC}{Curso en línea, masivo y abierto; por sus siglas en inglés, \textit{Massive Open Online Course}}
\newacronym{CAL}{CAL}{Aprendizaje asistido por ordenador, por sus siglas en inglés, \textit{Computer-Aided Learning}}
\newacronym{CAT}{CAT}{Tests adaptativos por ordenador, por sus siglas en inglés, \textit{Computer Adaptive Tests}}
\newacronym{WCAG}{WCAG}{Pautas para la accesibilidad del contenido web, por sus siglas en inglés, \textit{Web Content Accessibility Guidelines}. Es un estándar creado por la W3C}
\newacronym{SPOC}{SPOC}{Curso online pequeño y privado, por sus siglas en inglés, \textit{Small Private Online Course}. Término acuñado para referirse a una versión de un \acrshort{MOOC} usado localmente con estudiantes presenciales.}
\newacronym{e-valUAM}{e-valUAM}{Nombre con el que se conoce al sistema adaptativo para ayuda al aprendizaje desarrollado como parte de este Trabajo Fin de Grado.}
\newacronym{IRT}{IRT}{\textit{Item Response Theory}}
\newacronym{AH}{AH}{\textit{Adaptive hypermedia}}
\newacronym{AEH}{AEH}{\textit{Adaptive Educational Hypermedia}}
\newacronym{CTT}{CTT}{\textit{Classical Test Theory}}

% Glosario

% TODO: Añadir aquí las definiciones del glosario
% Ejemplo de glosario
\newglossaryentry{bitstream}{name={bitstream},description={En este contexto se refiere al binario que configura el Hardware de la FPGA}}


% Inicio del documento
\begin{document}

% Elección del idioma (español)
\selectlanguage{spanish}

%
% Portada
%
\pagenumbering{gobble}
%
% Portada
%

% Universidad, Facultad
\begin{titlepage}
\selectlanguage{spanish}
\begin{center}
\textbf{\begin{huge}
\universidad \\
\end{huge}}
\bigskip 
\begin{LARGE}
\facultad \\
\end{LARGE}
\end{center}

\bigskip
\bigskip

%
% Imágenes (logos) izquierdo y derecho
%
\begin{figure}[h]
	\begin{center}
		\includegraphics[scale=0.35]{\logoizq}
    	\hspace{1cm}
		\includegraphics[scale=0.4]{\logoder}
	\end{center}	
\end{figure}

\bigskip
\bigskip
\bigskip

% Grado
\begin{center}
\begin{large}
\textbf{\grado}\\
\end{large}
\end{center}

\bigskip

\textbf{\begin{center}
\begin{huge}
TRABAJO FIN DE GRADO
\end{huge}
\end{center}}

\bigskip
\bigskip

% Nombre del TFG
\begin{center}
\textbf{\begin{large}
\MakeUppercase{\nombretrabajo}\\
\end{large}}
\end{center}

% Nombre del autor...
\vspace{\fill}
\begin{center}
\textbf{\nombreautor}\\
%t ...tutor...
\textbf{Tutor: \nombretutor}\\
% ... y ponente, si está definido en main.tex
\ifcsname nombreponente\endcsname
\textbf{Ponente: \nombreponente}\\
\fi

\bigskip

% Fecha
\textbf{\MakeUppercase{\fecha}}\\
\end{center}
\end{titlepage}


\hypersetup{pageanchor=true}

%
% Resumen
%
\pagenumbering{Roman}
\setcounter{page}{0}
% Resumen en inglés
\chapter*{Abstract}

\begin{abstractEn}
The introduction of technologies in teaching-learning environments is a possibility that brings many promising hopes. The areas where technology can contribute to education are very wide, and so, this work is focused in the evaluation process. Based on the principles of the \textit{Computer Adaptive Tests} (\acrshort {CAT}) and the \textit {Adaptive Educational Hypermedia} (\acrshort {AEH}) as a result of this Final Project an online and adaptive system for asistence learning process is presented: e-valUAM.

The development of e-valUAM has aspired to achieve the following objectives:
\begin{enumerate*}[label=\alph*\upshape)]
\item Allow teachers, in any discipline, to create adaptive online questionnaires that serve to the teachers or the students themselves to measure their knowledge in a way challenges each student;
\item that is not necessary to have advanced knowledge in computer to respond, or create questionnaires;
\item make available to the teacher automatic analysis tools to facilitate its work of identifying deficiencies in the evaluation process; and 
\item implement a robust, flexible system that properly abstracts the dynamics of evaluation, so it can be used in real environments.
\end{enumerate*}

Throughout the life of this project has been created two functional and incremental prototypes that have been used for the academic years 2013/2014 and 2014/2015. To create both prototypes it has been conducted a complete analysis of requirements that has collected the characteristics and needs that should cover the system. Then there has been a design that has created domain, user and adaptation models that have been implemented after in e-valUAM.

Both prototypes have been successfully tested in real environments which have encompassed diverse disciplines. Thanks to this tool, students have been available to self-assessment questionnaires that have provided them with early feedback, teachers have been able to create automatic correction tests in which errors have been detected quickly and a group of researchers has been able to test new models of adaptive testing.
\end{abstractEn}

% Palabras clave en inglés
\begin{keywordsEn}
Computer Adaptive Tests, Adaptive Educational Hypermedia, adaptive systems, online systems, evaluation methodologies, teaching-learning environmentes.
\end{keywordsEn}

% Resumen en español
\chapter*{Resumen}

\begin{abstractEs}
La introducción de las tecnologías en los procesos de enseñanza-aprendizaje es una posibilidad que trae consigo muchas y prometedoras esperanzas. Las áreas donde la tecnología puede aportar a la educación son muy amplias, por lo que este trabajo se centra en el proceso de evaluación. Basándose en los principios de los \textit{Computer Adaptive Tests} (\acrshort{CAT}) y la \textit{Adaptive Educational Hypermedia} (\acrshort{AEH}), como resultado de este Trabajo Fin de Grado se presenta un sistema de ayuda al aprendizaje online y adaptativo: e-valUAM.

Con la creación de e-valUAM se ha aspirado a alcanzar los siguientes objetivos: 
\begin{enumerate*}[label=\alph*\upshape)]
\item Permitir crear a profesores de cualquier disciplina cuestionarios online adaptativos que sirvan para que los profesores o los propios alumnos puedan medir sus conocimientos de una forma que resulte un reto justo a cada alumno; 
\item que no sea necesario disponer de un conocimiento avanzado en informática para responder o crear los cuestionarios;
\item poner a disposición del profesor herramientas automáticas de análisis que faciliten su labor de detección de deficiencias en el proceso de evaluación; y 
\item implementar un sistema robusto, flexible y que abstraiga correctamente las dinámicas de una evaluación, para que pueda ser utilizado en entornos reales.
\end{enumerate*}

A lo largo de la vida del proyecto se han creado dos prototipos funcionales e incrementales que han sido utilizados durante los años académicos de 2013/2014 y 2014/2015. Para crear ambos prototipos se ha llevado a cabo un completo análisis de requisitos que ha recogido las características y las necesidades que debía cubrir el sistema. A continuación, se ha realizado un diseño en el que se han creado modelos de dominio, usuario y adaptación que después han sido implementados en e-valUAM. 

Ambos prototipos han sido probados con éxito en entornos reales que han englobado disciplinas dispares. Gracias a esta herramienta, los alumnos han tenido a su disposición cuestionarios de autoevaluación que les han aportado retroalimentación temprana, los profesores han podido crear exámenes de correción automática en los que han detectado rapidamente errores y un conjunto de investigadores han podido probar nuevos modelos de cuestionarios adaptativos.

\end{abstractEs}

% Palabras clave en español
\begin{keywordsEs}
Computer Adaptive Tests, Adaptive Educational Hypermedia, sistemas adaptativos, sistemas online, metodologías de evaluación, entornos enseñanza-aprendizaje.
\end{keywordsEs}

%
% Agradecimientos
%
\chapter*{Agradecimientos}
~

Este trabajo no hubiera sido posible sin que hace tres años Sacha y Pilar me dejaran jugar con ellos. Quiero creer que gracias a ello ahora soy un poco menos ``yogurín''. Gracias a Sacha por brindarme esta oportunidad y haberme empujado hacia adelante estos años y a Pilar por su impagable consejo, ayuda y paciencia.

También quiero dar las gracias a Covadonga Sevilla, Francisco L. Borrego, Santiago Atrio y Simone Santini por haber accedido a utilizar e-valUAM en sus clases. Así mismo, gracias (o perdón) a todos los alumnos de las asignaturas de ``Historia Antigua I'' del Grado en Historia, ``El Entorno como Instrumento Educativo'' del Grado en Educación Infantil, ``Proyecto de Programación'' del Doble Grado en Informática y Matemáticas y ``Diseño y Análisis de Algoritmos'' del Grado y Doble Grado en Ingeniería Informática.

Gracias a mis compañeros de carrera, con los que estos cuatro años se han hecho mucho más divertidos: Álvaro, Dani, Diego, Euler, Isa, José, Julián, María, Miguel, Mónica y Rober. Gracias a los compañeros que traje de antes, aún conservo y tanto me han acompañado. Gracias también a todos los grandes profesores con los que he tenido el placer de aprender estos cuatro años.

Por último, gracias a mis padres y a mi increíble hermana Lucía.



% Cita
\begin{flushright}
\textit{%<<I often liked to play tricks on people when I was at MIT. One time, in mechanical drawing class, some joker picked up a French curve (a piece of plastic for drawing somooth curves--–a curly, funny-looking thing) and said ``I wonder if the curves on this thing have some special formula?''\\
%I thought for a moment and said ``Sure they do. The curces are very special curves. Lemme show ya,'' and I picked up my French curve and began to turn it slowly. ``The French curve is mado so that at the lowest point on each curve, no matter how you turn it, the tangent is horizontal.''\\
%All the guys in the class were holding their French curve up at different angles, holding their pencil up to it at the lowest point and laying it along, and discovering that, sure enough, the tangent is horizontal. They were all excited by this ``discovery''---even though they hal already gone through a certain amount of calculus and had already ``learned'' that the derivate (tangent) of the minumen (the lowest point) of \emph{any} curve is zero (horizontal). They didn't put two and two together. They didn't even know what they ``knew.''\\
\\<<I don`t know what's the matter with people: they don't learn by understanding; they learn by some other way---by rote or something. Their knowledge is so fragile!>>\\}
Richard P. Feynman. ``Surely You're Joking, Mr. Feynman!''
\end{flushright}

  


%
% Tabla de contenidos
%

\begingroup
\renewcommand{\cleardoublepage}{}
\renewcommand{\clearpage}{}
\tableofcontents
\listoftables
\listoffigures
\endgroup

\cleardoublepage

%
% Glosario
%
\printglossary[title=Glosario,toctitle=Glosario]
\cleardoublepage


% Estilo de párrafo de los capítulos
\setlength{\parskip}{0.75em}
\renewcommand{\baselinestretch}{1.25}
% Interlineado
\spacing{1.3}
% Numeración contenido
\pagenumbering{arabic}
\setcounter{page}{1}

%
% Introducción
%
\pagenumbering{arabic}
\setcounter{page}{1}
\chapter{Introducción \label{sec:introduccion}}

En esta sección se detallará qué ha motivado la realización de este Trabajo Fin de Grado. A continuación, se explicará en qué marco se ha llevado a cabo, así como el alcance del proyecto, especificando sus objetivos y sus límites. Por último, se expondrá  la estructura que sigue el resto del presente documento.

\section{Motivación}

%\textbf{¿Por qué son necesarios los test adaptativos?
%¿Qué puede aportar un sistema informático a los test adaptativos?
%¿Dónde se pueden utilizar?
%Listar ejemplos: MOOCs, AEH, educación clásica, contextos donde el usuario no dispone de conocimiento informático.}

La revolución que ha supuesto la introducción de las tecnologías informáticas en cada día más aspectos de la vida humana es una revolución de un profundo calado. La informática ha traído consigo mejoras inconmensurables en las comunicaciones, la automatización o el desarrollo científico (por citar solo algunos ejemplos) que, en general, han permitido al ser humano librarse de tareas repetitivas y dedicar más esfuerzo a las tareas verdaderamente interesantes.

La educación, a pesar de ser uno de los pilares sociales, ha bebido de los avances en informática, pero a un ritmo mucho menor que otras áreas. Aunque cada día es más habitual el uso de ordenadores en las aulas, la proyección de diapositivas o el uso de pizarras digitales, gran parte de la actividad educativa ha permanecido inalterada, anclada en modelos artesanales. Por ejemplo, la evaluación de los alumnos a día de hoy se sigue basando principalmente en cuestionarios creados cada año, que los alumnos responden en papel y los profesores corrigen a mano, uno a uno.

El ejemplo de la evaluación de los alumnos trae consigo varios problemas. Por un lado, al crearse cada año un nuevo examen es probable que algún año haya alguna pregunta mal planteada que no se descubra hasta la corrección. Así mismo, la cantidad ingente de tiempo que pierden los profesores es tiempo que no dedican a explicar a los alumnos temario, reforzar las partes más complicadas o a responder dudas. Ese tiempo también se traduce en que los alumnos tardan tiempo en tener una retroalimentación sobre su empeño, una información que es muy valiosa y de la que cuanto antes dispongan, mejor.



%Algunas nuevas invenciones, como los \acrshort{MOOC} demuestran que el aprendizaje asistido por tecnologías, \acrshort{TEL}, es un área donde aún queda trabajo y en múltiples facetas. Este Trabajo 

%Desde su creación, los ordenadores han sido introducidos de forma progresiva en cada vez más sectores, con grandes beneficios. La educación es un ejemplo de ello, aunque aún todavía es un ejemplo incompleto. El aprendizaje asistido por tecnologías, \acrshort{TEL}, y en concreto, el aprendizaje asistido por ordenador, \acrshort{CAL}, es cada vez más habitual y ha sido aplicada con éxito a la educación presencial, semipresencial o a distancia, aportando grandes ventajas en cada modalidad. La reciente aparición y popularización de los \acrshort{MOOC} ha vuelto a demostrar la necesidad de seguir ampliando estas áreas {CITA}.

%Dentro de las \acrshort{CAL} una de las ramas de interés es la conocida como tests adaptativos por ordenador, o \acrshort{CAT}. Los \acrshort{CAT} se han utilizado para múltiples propósitos, como puntuación instantánea \cite{Wainer00}, la mejora de  competencias lingüísticas \cite{Chapelle06} , identificación de estilos de aprendizaje \cite{Ortigosa10}, la habilidad matemática \cite{Klinkenberg11}, o la evaluación del estado de salud \cite{Revicki97}.

%\textbf{¿Cuál ha sido exactamente el trabajo? ¿Motivación? Creo que no ¿e-valUAM? Sí ¿Modelo de estudiantes? Sí ¿Protocolo para crear las preguntas? También ¿Exámenes con duda? Sí.}

% sirve para múltiples objetivos, como aumentar la motivación de los alumnos {CITA}, sus resultados académicos {CITA} o su }}

\section{Marco}

En qué investigación se engloba el proyecto. ¿Citar financiación?

\section{Alcance y objetivos}

¿Qué pretende lograr el sistema?
¿Qué NO pretende lograr el sistema?

\section{Estructura del documento}

TODO: Descripción de la estructura del documento

% Cuatro líneas. Resumir todo el trabajo. Ir haciéndolo
%
% Estado del arte
%
\chapter{Estado del Arte\label{sec:estado_del_arte}}

TODO: Estado del arte

DEJAR PARA EL FINAL

AEH
MOOC




Computer-assisted assessment with item classification
for programming skills

The use of computers has been extended to almost any discipline
nowadays. Being essential for the students in the performance of
their future professional activities, it is also more than adequate to
include the use of computers in any dimension of their learning
process. One of the topics that can be strongly improved by
computer-assisted tools is the assessment process. It is apparent
that, at least for certain purposes, computer-based testing (CBT)
can offer distinct advantages, apart from instantaneous scoring
[1]. Within CBT, Computer Adaptive Testing (CAT) has been
used for very different purposes; such as language proficiency
improvement [2], learning styles identification [3], measurement
of chess playing proficiency [4] or Maths ability [5], improvement
of the efficiency of personality [6] or heath status assessment [7].
Different versions of CAT have been used in learning
environments for decades in order to personalize the learning
process [8][9][10][11]. In that framework, CAT allows seeing the
students as individuals, taking their own characteristics into
account. Typically, CAT systems are able to adapt the items
presented to the student depending on their former answers, often
including some kind of personalized feedback [12].
Though presenting evident advantages, CAT systems also face
some clear difficulties, been one of them the need of precalibrating
the items of the test before they can be used in real
situations. This implies that the test should be performed by a
large sample of the population in order to assume a good test
calibration [5]. However, such “real” pre-calibration is not always
feasible, what has led to a huge interest in the development of
different kinds of students’ models which could help in the precalibration
task [13][14][15][16].

Adaptive Model for Computer-Assisted Assessment in Programming Skills

Being essential for the students’ performance of their potential professional activities, it is more than adequate to consider the
use of computers in any dimension of their learning process. One of the topics that can be strongly improved by computerassisted
tools is the assessment process. In this context, Computer-based testing (CBT) can offer distinct advantages. Within
CBT, Computer Adaptive Testing (CAT) has been used for very different purposes such as instantaneous scoring [1],
language proficiency improvement [2], learning styles identification [3], measurement of chess playing proficiency [4], Maths
ability [5], improvement of the efficiency of personality [6], or heath status assessment [7]. Different versions of CAT have
been used in learning environments in order to personalize the learning process [8] [9] [10] [11]. In that framework, CAT
allows seeing the students as individuals, taking their own characteristics into account. Typically, CAT systems are able to
adapt the items presented to the student depending on their former answers, often including some kind of personalized
feedback [12].
One of the most important difficulties in CAT systems is the need of pre-calibrating the items before they are used in real
assessments. For this reason, the test should be performed by a large sample of the population in order to get a good test
calibration [5]. The complexity of pre-calibration has led to a huge interest in the development of students’ models which
could help in the pre-calibration task [13] [14] [15] [16].


Constructing quality test with e-valUAM

Technology-Enhanced Learning (TEL) has a growing presence in educational institutions by making use of Computer-Aided Learning (CAL): a keynote practice where teachers and students are assisted by computers in their teaching/learning processes. Under this perspective, the assessment process is one of those issues that can get profit from the use of these practices. Moreover, completely on-line learning environments, like the emergent trends in Massive Open Online Courses (MOOCs) [1], demonstrate the importance of using CAL environments.

In this context, learning systems, which provide any kind of immediate feedback to students, look to be more effective than traditional learning strategies [2]. We can find in the literature several examples of assessment tools for some specific learning domains that can give useful and interactive advices to the students [3] [4] [5] [6]. In the context of providing feedback and advice to the students for widespread domains, Computer Adaptive Testing (CAT) provides advantages apart from an instantaneous scoring [7]. CAT takes the students' characteristics into account in order to be able to adapt the items depending on their former answers, often including some kind of personalized feedback [8]. CAT applications can be found in language proficiency improvement [9], learning styles identification [10], measurement of chess playing proficiency [11], Maths ability [12], and improvement of the efficiency of personality [13] or heath status assessment [14]. Different CAT approaches has been also used in learning environments for decades adapting the learning process [15] [16] [17] [18].

In spite of the advantages of this methodology, items can be inadequately developed by being wrongly classified on a topic or not clearly written up, which can lead to poor quality tests. This situation leads to face an important issue: the need of pre-calibrating test items before they can be used. Usually, a good calibration process implies that the test should be performed by a large population sample [12]. Nevertheless, since the pre-calibration is not always feasible, student models are applied in order to refine the quality of the assessment strategy [19] [20] [21] [22].

%
% Diseño y desarrollo
%
\chapter{Diseño y desarrollo\label{sec:disenhoYDesarrollo}}

\textbf{Rápida intro.}

TODO: Demostrar todo el dominio que pueda sobre cuestiones de la carrera.

\section{Diseño}

\subsection{Análisis funcional}

\textbf{¿Detallar un análisis funcional? Sí}

\subsection{Modelo de usuario}

\textbf{Profesor y alumno}

\subsection{Modelo del dominio}

\textbf{Estructura de la BD}

\subsection{Modelo de adaptación}

\textbf{Exámenes con distintos niveles}

\subsection{Motor de adaptación}

\textbf{????}

\section{e-valUAM}



%
% Pruebas y resultados
%
\chapter{Pruebas y resultados\label{sec:pruebasYResultados}}

% IDEAS A VENDER:
% Generalizable y escalable. Podemos cambiar el modelo subyacente sin sufrir
% Meter el apartado de análisis de resultados
% Se han usado por 50 personas y ha aguantado
% Hablar de la base de datos. Del diseño y cómo se ha hecho
% Probado en entorno real

\section{2013-2014: Primer prototipo}

% Añadir la parte de análsis de datos

Durante el curso académico 2013-2014, 15 alumnos de la asignatura de ``Historia Antigua I'' en el Grado en Historia de la Universidad Autónoma de Madrid utilizaron e-valUAM como herramienta de estudio y de evaluación. Fue el entorno donde se probó el primer prototipo, el cual ya permitía responder cuestionarios, pero no permitía a los profesores crearlos de forma autónoma ni les ofrecía posibilidades multimedia avanzadas (solo se permitía una imagen opcional por pregunta). De esta forma, la experiencia se centró en probar la experiencia de uso de los alumnos y la robustez del sistema para responder en paralelo a todas las peticiones.

\begin{figure}[htp!]
	\centering
	\includegraphics[width=0.75\textwidth,clip=true]{e-valUAM_primera_version}
	\caption{Interfaz del módulo del examen del primer prototipo}
	\label{fig:e-valUAM primera version}
\end{figure}

Desde el mes de octubre hasta el final del curso, los alumnos tuvieron disponibles 3 cuestionarios de auto evaluación y realizaron 4 exámenes. La auto evaluación estaba disponible a los alumnos tantas veces como quisieran. De los 15 alumnos, 6 utilizaron la aplicación para auto evaluación menos de 3 veces, mientras que otros 6 la utilizaron más de 15 veces. Los exámenes fueron dos parciales y un examen final para la convocatoria ordinaria y otra para la extraordinaria, al que solo se tuvieron que presentar 3 alumnos. La calificación determinada por la aplicación se ponderó en la nota final de la asignatura. A lo largo del año, entre tres profesores de Historia crearon un total de 372 preguntas, respondiéndose cada una de ellas 12,386 veces de media. 

Los resultados de la experiencia fueron muy satisfactorios. A nivel técnico, el sistema respondió como se esperaba durante todo el proceso, sin sufrir ningún tipo de caída. La mayor prueba de estrés del sistema fue el día del examen final. En las tres horas previas al examen se realizaron 31 accesos a los cuestionarios de auto evaluación, lo que supuso aproximadamente unas 300 peticiones al servidor. Cuando empezó el examen, los 15 alumnos lo realizaron a la vez, lo que supuso aproximadamente 68 peticiones por minuto al servidor durante los 20 minutos que duró el examen. El sistema logró almacenar todas las respuestas correctamente, seleccionar todas las preguntas siguientes siguiendo el modelo y ningún alumno tuvo que esperar entre preguntas ni sufrió ningún corte del servicio. Tampoco se registró ningún problema en los 3 meses que los alumnos hicieron un uso más intensivo del mismo (de noviembre de 2013 a enero de 2014).

% Hablar de qué se decidió cambiar
La experiencia de los alumnos con el sistema fue en general satisfactoria, aunque algunos mostraron malestar con el modelo del examen. Las mayores molestias venían provocadas porque no se pudieran dejar preguntas en blanco ni que se pudiera revisar una respuesta anterior. Aunque la segunda es una imposición del modelo (al depender las nuevas preguntas de las respuestas anteriores, estas no pueden cambiar), la primera sí se tuvo en consideración añadiendo la opción al modelo de la respuesta con duda.

De cara a la creación del segundo prototipo, además de añadir la respuesta con duda, se decidió crear el apartado de gestión del profesor, además de aumentar las capacidades de la aplicación para trabajar con ficheros multimedia. Así mismo, se hizo una actualización de la interfaz para incorporar un diseño responsive y adaptadarlo a toda las nuevas posibilidades multimedia que iban a incorporarse.

\section{2014/2015: Segundo prototipo}

Durante el curso académico 2014/2015 se utilizó en la asignatura de ``El Entorno como Instrumento Educativo'' del Grado en Educación Infantil. Las pruebas realizadas en este entorno fueron mucho más meticulosas que en el curso anterior, por varios motivos. 

Primero, los alumnos eran más y por lo tanto la aplicación se enfrentó a mayores picos de demanda. Segundo, se añadió más funcionalidad tanto para los profesores como para los alumnos. Tercero, los docentes utilizaron la aplicación como asistencia en su labor, pero además utilizamos la aplicación para conocer el nivel de conocimientos informáticos del que disponían los alumnos y conocer así mejor a los usuarios que iban a probar el sistema.



% Contar aquí que cambiamos el modelo

\subsection{Test de conocimientos informáticos}

\subsection{Alumnos del Grado en Educación Infantil}

\subsubsection{Aprendizaje}
\subsubsection{Examen final}
\subsubsection{Trabajo final}

\section{Comparativa con otras soluciones}

%
% Conclusiones
%
\chapter{Conclusiones\label{sec:conclusiones}}

A modo de conclusión, a continuación se presentan los \textbf{logros más relevantes} que ha alcanzado el proyecto expuesto a lo largo de todo el documento, con los que se logran satisfacer todos los objetivos que se recogieron en la introducción:

\begin{itemize}
	\item Se ha diseñado un \textbf{sistema adaptativo} inspirado en \textit{Computerized Adaptive Testing} y la \textit{Adaptive Educational Hypermedia} que, utilizando información de modelos de usuarios, de dominio y de adaptación, permite crear cuestionarios personalizados que profesores y alumnos pueden utilizar para medir su rendimiento académico recibiendo de forma instantánea retroalimentación.
	\item Se han desarrollado las herramientas necesarias para que los profesores puedan crear las baterías de preguntas a través de una interfaz online que les abstrae de los modelos subyacentes logrando así que pueda ser utilizada por \textbf{usuarios sin conocimientos informáticos} avanzados.
	\item Se ha diseñado un sistema capaz de trabajar con \textbf{ficheros multimedia} que ofrecen más posibilidades a los profesores a la hora de crear las preguntas. Así mismo, se ha diseñado una base de datos que ha logrado almacenar toda la información sobre el uso, permitiendo análisis posteriores que han llevado a la creación de nuevos modelos y publicaciones científicas.
	\item Se han implementado mecanismos de \textbf{análisis automático} de resultados que permiten detectar a los profesores preguntas defectuosas de forma instantánea para que puedan corregirlas o eliminarlas, simplificando una tarea muy costosa.
	\item Se han construido \textbf{dos prototipos} incrementales y funcionales durante dos años académicos en los cuales han demostrado su flexibilidad, al haber sido probado en ellos varios modelos de adaptación sin que ello haya significado remodelaciones en el resto de módulos.
	\item Ambos prototipos han sido utilizados con éxito en\textbf{ dos entornos reales}, en dos disciplinas distintas y con más de 100 alumnos, habiendo situaciones donde 50 alumnos han utilizado en paralelo el sistema sin que el rendimiento bajara.
\end{itemize}

Por otra parte, aunque todos los objetivos se han cumplido, algunos, como el relativo al análisis de los resultados, podrían beneficiarse de más desarrollo. Así mismo, a lo largo del desarrollo del proyecto han ido surgiendo ideas y propuestas de mejoras del sistema y con todo ello se ha elaborado una lista de retos que desean abordarse en un futuro cercano:

\begin{itemize}
	\item Añadir \textbf{más análisis de resultados}, como remarcar preguntas que tienen un tiempo de respuesta más alto o crear informes con la evolución de cada alumno a lo largo del año o el conjunto agrado de toda la clase. Así mismo, también se desea añadir una biblioteca de gráficos en Javascript (como D3.js) para \textbf{mejorar la visualización} de todos los análisis.
	\item Permitir que se puedan crear \textbf{nuevos tipos de pregunta}, como preguntas de respuesta abierta o preguntas generadas en función a un código especificado por el profesor.
	\item Dar la opción a los profesores de \textbf{elegir qué modelo} de adaptación desean utilizar entre todos los desarrollados.
	\item \textbf{Ampliar} las opciones que ofrecen las herramientas de creación y gestión de materias, preguntas y cuestionarios que tienen accesibles los profesores.
	\item Dedicar esfuerzos en la \textbf{interacción persona-ordenador}. Para ello, se pretende recolectar información sistemática mediante cuestionarios sobre la experiencia de uso a los alumnos que ya lo han probado para solucionar las deficiencias que puedan existir y luego centrar los esfuerzos a hacer que el sistema sea \textbf{accesible} para usuarios con diversidad funcional.
	\item Facilitar las \textbf{tareas de administración} mediante una interfaz web.
\end{itemize}
 

%
% Página en blanco
%
\cleardoublepage

\addtocontents{toc}{\protect\newpage}

%
% Bibliografía
%
\bibliography{src/bibliografia}
\addcontentsline{toc}{chapter}{Bibliografía}

% No expandir elementos para llenar toda la página
\raggedbottom

%
% Apéndices
%
\appendix
\cleardoublepage
\addappheadtotoc
\appendixpage

%
% TODO: Apéndices del TFG
%
% \chapter{Análisis de requisito ampliado\label{apen:analisis de requisitos}}

% \begin{rnf0}
% 	\item Accesibilidad
% 	\begin{rnf0*}
% 		\item El sistema debe cumplir con el estándar \acrshort{WCAG} 2.0 en un nivel A.
% 	\end{rnf0*}
% \end{rnf0}


\chapter{Estructura del proyecto\label{apen:estructura proyecto}}

\dirtree{%
 .1 /.
 .2 index.php\DTcomment{Pantalla de login}.
 .2 eleccionExamen.php\DTcomment{Presentación y listado de exámenes disponibles}.
 .2 examen.php\DTcomment{Fichero con el motor de adaptación}.
 .2 finExamen.php\DTcomment{Resultados del examen}.
 .2 cambiarContrasenya.php.
 .2 estilo.css.
 .2 favicon.png.
 .2 funcionesAlumno.php\DTcomment{Fichero auxiliar con funciones PHP}.
 .2 multimedia\DTcomment{Directorio para los ficheros multimedia}.
 .3 0\DTcomment{Un directorio por cada materia, con sus ficheros dentro}.
 .4 Amon.bmp.
 .4 Anubis.bmp.
 .4 Asedioramses.bmp.
 .4 \ldots.
 .3 1.
 .4 Archivoebla.bmp.
 .4 Codigodehammurabi.bmp.
 .4 \ldots.
 .3 \ldots.
 .3 46\DTcomment{En total se han creado 46 materias distintas}.
 .4 1.mp3.
 .4 1a.jpg.
 .4 1b.jpg.
 .4 1c.jpg.
 .4 2.mp3.
 .4 \ldots.
 .3 logos\DTcomment{Directorio con los logos corporativos}.
 .4 ope.bmp.
 .4 uam.jpg.
 .2 profesor\DTcomment{Directorio con los ficheros de la sección de profesor}.
 .3 index.php\DTcomment{Login}.
 .3 salir.php\DTcomment{Logout}.
 .3 ayuda.php\DTcomment{Introducción al sistema de gestión}.
 .3 profesor.php\DTcomment{Comprueba que el usuario tenga permisos de profesor}.
 .3 funcionesProfesor.php\DTcomment{Fichero auxiliar con funciones PHP}.
 .3 materiasGestion.php\DTcomment{\texttt{*Gestion} es la interfaz de creación y listado}.
 .3 materaiasRequest.php\DTcomment{\texttt{*Request} procesa peticiones AJAX de \texttt{*Gestion}}.
 .3 preguntasGestion.php\DTcomment{Hay un \texttt{*Gestion} y un \texttt{*Request} para cada ente}.
 .3 preguntasRequest.php.
 .3 preguntasMostrar.php\DTcomment{Interfaz que lista todas las preguntas de una materia}.
}

\dirtree{%
 .1 /.
 .2 profesor.
 .3 examenGestion.php.
 .3 examenRequest.php.
 .3 multimediaGestion.php.
 .3 multimediaRequest.php.
 .3 multimediaView.php\DTcomment{Interfaz que lista todos los ficheros de una materia}.
 .3 estadisticas.php\DTcomment{Módulo de análisis de resultados}.
 .3 estadisticasRequest.php.
 .3 recuperarExamenesRequest.php\DTcomment{Módulo para revisar los cuestionarios respondidos}.
 .3 recuperarExamenes.php.
 .2 vendor\DTcomment{Directorio con las bibliotecas usadas en el proyecto}.
 .3 bootstrap.min.css.
 .3 bootstrap.min.js.
 .3 boostrap-confirmation.js.
 .3 jquery.min.js.
}

%\chapter{Código más relevante\label{apen:codigo}}


%
%\chapter{Ejemplos de bloques y comandos útiles en LaTeX\label{sec:ejemplos}}
%\section{Ejemplo de sección}
%
%
% Breve guía de comandos útiles para la memoria
%
%
% Citar una referencia
%La DARPA creo el protocolo de Internet \cite{ipv4sta}.
%
% Citar un elemento del glosario
%Citamos el acrónimo \gls{FPGA}.
%
% Citar un elemento del glosario (primera letra en may´usculas)
%\Gls{bitstream} es una secuencia de bits.
%
% Insertar una imagen con pie de página
%\begin{figure}[htp!]
%  \centering
%  \includegraphics[width=0.75\textwidth,clip=true]{Logo_UAM}
%  \caption{Logo de la Universidad Autónoma de madrid.}
%  \label{fig:logo_uam}
%\end{figure} 
%
% Referenciar una etiqueta (label)
%La figura~\ref{fig:logo_uam} se utiliza en la portada.
%
% Nueva página
%\clearpage
%
% Añadir código fuente sin líneas
%\begin{lstlisting}[label=algoritmo:quicksort,language=C,frame=single,caption=Algoritmo de ordenación Quicksort]
%#include <stdio.h>
% 
%void quick_sort (int *a, int n) {
%    int i, j, p, t;
%    if (n < 2)
%        return;
%    p = a[n / 2];
%    for (i = 0, j = n - 1;; i++, j--) {
%        while (a[i] < p)
%            i++;
%        while (p < a[j])
%            j--;
%        if (i >= j)
%            break;
%        t = a[i];
%        a[i] = a[j];
%        a[j] = t;
%    }
%    quick_sort(a, i);
%    quick_sort(a + i, n - i);
%}
%\end{lstlisting}
%
% Bloque de código inseparable
%\begin{code}
%#include <stdio.h>
% 
%void quick_sort (int *a, int n) {
%    int i, j, p, t;
%    if (n < 2)
%        return;
%    p = a[n / 2];
%    for (i = 0, j = n - 1;; i++, j--) {
%        while (a[i] < p)
%            i++;
%        while (p < a[j])
%            j--;
%        if (i >= j)
%            break;
%        t = a[i];
%        a[i] = a[j];
%        a[j] = t;
%    }
%    quick_sort(a, i);
%    quick_sort(a + i, n - i);
%}
%\end{code}
%
% Fórmula dentro de una línea de texto
%La ecuación de Euler ($e^{ \pm i\theta } = \cos \theta \pm i\sin \theta$) es citada frecuentemente como un ejemplo de belleza matemática.
%
% Fórmula independiente
%\begin{equation}\label{eq:pythagoras}
%a^2 + b^2 = c^2
%\end{equation}


% Fin del documento
\end{document}
